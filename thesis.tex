% Vorlage für Masterarbeiten
%	Konfiguration in texmaker: Options -> Configure Texmaker -> Quick Build -> Select Latexmk + ViewPDF
%	Entsprechende Informationen in den config/metainfo verändern
%   Beispieldokument entsteht bei Kompilierung nach oben angegebener Reihenfolge
\RequirePackage[hyphens]{url}
\documentclass[
   fontsize=12pt,                % Schriftgroesse 12pt
   paper=a4,             % Layout fuer Din A4
   %oneside,            % einseitig
   twoside,             % Layout fuer beidseitigen Druck
   %openright,          % Start auf rechter Seite
   headinclude,         % Kopfzeile wird Seiten-Layouts mit beruecksichtigt
   headsepline,         % horizontale Linie unter Kolumnentitel
	%footinclude,
	%footexclude,		%fusszeile nicht bercksichtigen
   %plainheadsepline,    % horizontale Linie auch beim plain-Style
   BCOR=12mm,            % Korrektur fuer die Bindung
   DIV=14,               % DIV-Wert fuer die Erstellung des Satzspiegels, siehe scrguide
   %DIVcalc,							% automatische Berechnung des Satzspiegels
   %openany,             % Kapitel knnen auf geraden und ungeraden Seiten beginnen
   %openright,							% Kapitel auf rechten Seiten beginnen!
   %pointlessnumbers,    % Kapitelnummern immer ohne Punkt
   fleqn,               % fleqn: Glgen links (statt mittig)
   %draft,               % Keine Bilder in der Anzeige, overfull hboxes werden angezeigt
   appendixprefix,				%berschriften des Anhangs +"Anhang"
  %chapterprefix,				%berschriften der Kapitel +"KAPITEL"
  %tablecaptionabove,    %tabellen immer als berschriften
   abstract=true,
   pagesize=pdftex						%berschrift Zusammenfassung einschalten
     ]{scrreprt}

%===============================================================================
% zentrale Layout-Angaben und Befehle
%===============================================================================
\newcommand\meta{./meta}
\input{\meta/config/commands}
%===============================================================================
% LATEX-Dokument
%===============================================================================

\usepackage[acronym,toc]{glossaries}

\makeglossaries

\input{\meta/config/hyphenation}
\begin{document}
%===============================================================================
% Zum Kompilieren pdflatex und bibtex ausführen.
% Konfiguration in texmaker: Options -> Configure Texmaker -> Quick Build -> Select Latexmk + ViewPDF
% Entsprechende Informationen in den config/metainfo verändern
% Zur Auswahl der Sprache im folgenden Befehl
% ngerman für deutsch eintragen, english für Englisch.
%===============================================================================
\selectlanguage{ngerman}
\setstretch{1.1}
%% Titelseite
\maketitle

%% Seitenlayout
\pagestyle{scrplain}
\cleardoubleemptypage
\pagenumbering{Roman}
\tableofcontents
\newpage
\listoffigures
\newpage
\listoftables
\newpage
\listofalgorithms
\newpage
%%   Falls nur Abkuerzungs- oder Symbolverzeichnis benoetigt wird, folgende Befehle benutzen   %%
%\printnomenclature

%%   Abkuerzungs- und Symbolverzeichnis   %%
%%   Zum alphabetischen Sortieren in der Konsole "makeindex filename.nlo -s nomencl.ist -o filename.nls" ausfuehren; nach dem zweiten Kompilieren erscheint das Abkuerzungsverzeichnis dann im Dokument   %%
%%   Alternativ im TeXnicCenter unter Ausgabe - Ausgabeprofile definieren - Nachbearbeitung einen Postprocessor ("Neu"-Button oben rechts) erstellen. Nach Bennenung unten als Anwendung Makeindex (aus Verzeichnis "./miktex/bin/makeindex.exe") auswaehlen und als Argumente folgenden String setzen: "%bm".nlo -s nomencl.ist -o "%bm".nls"   %%
%\usetwonomenclatures
%\printnomenclature

% Glossar und Akronymverzeichnis
\printglossaries

\cleardoubleemptypage
\pagestyle{scrheadings}
\pagenumbering{arabic}
\setcounter{page}{1}
% 1.5-facher Zeilenabstand für die Arbeit
\setstretch{1.5}

%===============================================================================
% LATEX-Dokument: Kapitel laden
%===============================================================================
%
% hier einzelne Kapitel mit \input{Kapitel-File} einfügen
%
\ifgit
\input{\meta/exampleContent/version}
\fi
\input{\meta/exampleContent/exampleContentThesis}


\cleardoubleemptypage
\pagestyle{scrplain}
% 1.1-facher Zeilenabstand für das Literaturverzeichnis und den Anhang
\setstretch{1.1}
\bibliographystyle{IEEEtranSN}%alphadin}%alpha}%abbrvdin}%natdin}%plaindin}%apa}
%plaindin = Literaturverzeichnis nach deutschem DIN-Standard siehe Webseite von Lorenzen www.haw-hamburg.de/pers/Lorenzen
\bibliography{\meta/exampleLiterature/bib}

\cleardoubleemptypage%
% Stichwortverzeichnis soll im Inhaltsverzeichnis auftauchen
% Sprungmarke mit Phantomsection korrigiert
\phantomsection%
\addcontentsline{toc}{chapter}{Index}%
% Stichwortverzeichnis endgueltig anzeigen
\printindex%

\appendixtitletocon
\begin{appendices}
\setstretch{1.1}
\chapter{Anhang}
\end{appendices}

\cleardoubleemptypage
\erklaerung

\end{document}
